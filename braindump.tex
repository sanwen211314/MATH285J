
\documentclass[12pt]{article}
\usepackage[top=1 in, bottom=1 in, left=1 in, right = 1 in]{geometry}
\usepackage{amsmath}
\usepackage{amsthm}
\usepackage{amssymb}
\usepackage{graphicx}

\newcommand{\Corr}{\text{Corr}}



\begin{document}
If we observe enough of the entries (80\% in this example), we get perfect reconstruction for sparse matrices with relatively high probability (perfect reconstruction roughly half the time) using nuclear norm minimization $$\min \| X \|_* \,\,\,\, \text{subject to} \,\,\,\, P_\Omega(X) = P_{\Omega}(M).$$ 

\begin{center}
\includegraphics[width=\textwidth]{sparseReconstruction.png}\\
\end{center}

\noindent\makebox[\linewidth]{\rule{\textwidth}{2pt}}

When we sample much less (30\% in this example), we no longer get perfect reconstruction, but we do preserve some of the structure of the matrix. Note the entries circled in red: these are large entries in the original matrix and, though they are unobserved, the reconstruction matches them fairly well. [Of course, the reconstruction also misses some large unobserved entries.
  
\begin{center}
\includegraphics[width=\textwidth]{sparseReconstructionLowObv.png}\\
\end{center}  
  
  
\noindent\makebox[\linewidth]{\rule{\textwidth}{2pt}}  

Moving to dense matrices, we cannot expect perfect reconstruction even when observing 80\% of entries.
  
\begin{center}
\includegraphics[width=\textwidth]{denseReconstruction.png}\\
\end{center}  
  
\noindent\makebox[\linewidth]{\rule{\textwidth}{2pt}}  

We want to consider matrice with certain columns correlated. As a first attempt, we let the first ten columns be "perfectly correlated" (i.e., they are all constant multiples of a single vector so that their correlation matrix is a $10\times 10$ matrix where every entry is $1$). In this case, we can selectively sample the entries. Here we sample 30\% of the entire matrix (120 entries), but we take only 20\% (24 entries) of these samples from the correlated columns and the remaining 80\% (96 entries) from the uncorrelated columns. Note, we are still using the nuclear norm minimization $$\min \| X \|_* \,\,\,\, \text{subject to} \,\,\,\, P_\Omega(X) = P_{\Omega}(M).$$ This selective sampling does worse than if we just sample evenly in the same scenario. This should be expected since we haven't actually used the correlation yet.
  
 \begin{center}
\includegraphics[width=\textwidth]{selectSample1.png}\\
\end{center}  
  
\noindent\makebox[\linewidth]{\rule{\textwidth}{2pt}} 
  
 We tried to think of  ways to incorporate the correlation into the problem. A first thought might be to change the functional to incorporate the correlation matrix itself, perhaps: $$\min \| X \|_* + \gamma \| \Corr(X_\tau) + \Corr(M_\tau)\|^2_F \,\,\,\, \text{subject to} \,\,\,\, P_\Omega(X) = P_{\Omega}(M)$$ where $\tau$ represents the collection of columns that are known to be correlated. However, the map $X\mapsto \Corr(X)$ is non-linear and so that map $X\mapsto \| \Corr(X_\tau) - \Corr(M_\tau)\|_F^2$ is non-convex. If we wanted to do this, we would need some more sophisticated non-convex optimization routines. Instead, it may be better to use some portion of the samples to produce an estimation $\tilde M_\tau$ to $M_\tau$ and then insert a fidelity term in order to match $X_\tau$ to $\tilde M_\tau$ and combine this with nuclear norm minimization: $$\min \| X \|_* + \gamma \| X_\tau + \tilde M_\tau\|^2_F \,\,\,\, \text{subject to} \,\,\,\, P_\Omega(X) = P_{\Omega}(M).$$ This idea shows some promise. Indeed assuming we can produce $\tilde M_\tau$ with some amount of accuracy, we can sample at a much lower rate from the correlated columns since they are accounted for by the fidelity term. Indeed, we did a small parameter sweep and it suggested that taking $\gamma \tilde 1$ and taking all the samples from the uncorrelated columns is ideal (here we assume we have used 40 samples to construct $\tilde M_\tau$ and then take the remaining $80$ samples from the uncorrelated portion). 
 
  \begin{center}
\includegraphics[width=\textwidth]{selectSample1.png}\\
\end{center}  
  
\noindent\makebox[\linewidth]{\rule{\textwidth}{2pt}} 
  
Next we are focusing on how to accurately reconstruct $M_\tau$ with relatively few samples, given that we know that the columns are correlated (or an easier problems: given that we know exactly how the columns are correlated; i.e., we know $\Corr(M_\tau)$ exactly). 

We've found that singular value threshholding (SVT) works pretty well for large, low rank matrices so we may be able to use SVT for $\tilde M_\tau$ if we assume that $M_\tau$ is low-rank. In the following picture, we are using SVT to approximately reconstruct at rank 1, $300 \times 300$ matrix. We are still not using any information about the correlation (How to incorporate this into SVT?) but even so the reconstruction is very good when we are only observing 10\% of the entries.

  \begin{center}
\includegraphics[width=0.7\textwidth]{SVT.png}\\
\end{center}  
 
Otherwise, we may be able to use multivariate regression to identify the correlation. PLEASE WRITE A SMALL AMOUNT ABOUT THE MV REGRESSION
 
 
\end{document}